\documentclass[a4paper,11pt]{article}
\usepackage[utf8]{inputenc}
\usepackage[T1]{fontenc}
\usepackage[french]{babel}
\usepackage{geometry}
\usepackage{tikz}
\usepackage{float}
\usepackage{hyperref}
\usepackage{xcolor}
\usepackage{enumitem}
\usepackage[normalem]{ulem} % Pour le soulignement des clés primaires
\usetikzlibrary{shapes.geometric, arrows.meta, positioning, fit, shadows, calc}

% Configuration des marges pour maximiser l'espace (max 6 pages)
\geometry{hmargin=2cm,vmargin=2cm}

% Couleurs
\definecolor{primary}{RGB}{0, 50, 100}
\definecolor{secondary}{RGB}{245, 245, 245}

\title{\textbf{\Huge Rapport de Projet}\\ \vspace{0.2cm} \Large Système de Gestion de Pharmacie}
\author{Généré par Antigravity}
\date{\today}

\begin{document}

\maketitle
\vspace{-0.5cm}
\hrule
\vspace{0.5cm}

\section{Introduction}
Ce rapport présente la conception et la réalisation d'une application de gestion de pharmacie. Il détaille l'architecture technique, les modèles de données et les fonctionnalités offertes.


\section{Architecture Technique et Structure du Projet}
Le projet est construit en Java, en respectant une séparation stricte des responsabilités (MVC).

\subsection{Gestion de Projet avec Maven}
Le projet utilise \textbf{Maven} comme gestionnaire de construction et de dépendances. Le fichier \texttt{pom.xml} centralise la configuration, notamment :
\begin{itemize}
    \item \textbf{Gestion des dépendances} : Inclusion automatique des bibliothèques nécessaires comme \texttt{mariadb-java-client} pour la connexion base de données et les modules JavaFX (\texttt{javafx-controls}, \texttt{javafx-fxml}).
    \item \textbf{Cycle de vie} : Automatisation de la compilation et de l'exécution via les plugins Maven (\texttt{javafx-maven-plugin}).
\end{itemize}

\subsection{Organisation du Code Source}
Le code est structuré en paquets (packages) logiques pour faciliter la maintenance :
\begin{description}
    \item[\texttt{pharmacie.model}] Contient les classes (POJO) représentant les entités du système (ex: \texttt{Product}, \texttt{Sale}).
    \item[\texttt{pharmacie.dao}] (Data Access Object) Gère toutes les interactions directes avec la base de données (CRUD).
    \item[\texttt{pharmacie.service}] Contient la logique métier et fait le lien entre les contrôleurs et les DAO (ex: validation des stocks).
    \item[\texttt{pharmacie.view}] Contient les fichiers FXML définissant l'interface graphique.
    \item[\texttt{pharmacie.controller}] Gère les événements utilisateurs de l'interface graphique (clics, saisies).
    \item[\texttt{pharmacie.app}] Contient le point d'entrée de l'application (\texttt{Main}).
\end{description}

\section{Diagramme de Classe et Description}
L'architecture logicielle repose sur le modèle MVC (Modèle-Vue-Contrôleur) avec une couche DAO pour la persistance.

\subsection{Diagramme de Classes UML}
\begin{center}
\begin{tikzpicture}[
    scale=0.7, transform shape,
    class/.style={rectangle, draw=black, fill=white, text width=4cm, align=left, font=\scriptsize, drop shadow},
    header/.style={draw=black, fill=gray!20, text width=4cm, align=center, font=\bfseries\scriptsize, minimum height=0.5cm},
    arrow/.style={thick, -{Latex[length=2mm]}},
    composition/.style={thick, {Diamond[fill=black]}-}
]

% Nodes
\node[header] (UserH) {User};
\node[class, below=0pt of UserH] (User) {+ id: int\\+ username: String\\+ role: Role};

\node[header] (SaleH) [right=of UserH, xshift=1cm] {Sale};
\node[class, below=0pt of SaleH] (Sale) {+ id: int\\+ total: double\\+ items: List<SaleItem>\\+ registerSale()};

\node[header] (ItemH) [right=of SaleH, xshift=1cm] {SaleItem};
\node[class, below=0pt of ItemH] (Item) {+ quantity: int\\+ price: double\\+ getTotal(): double};

\node[header] (ProdH) [below=of ItemH, yshift=-0.5cm] {Product};
\node[class, below=0pt of ProdH] (Prod) {+ name: String\\+ stock: int\\+ price: double};

\node[header] (ClientH) [below=of SaleH, yshift=-1.5cm] {Client};
\node[class, below=0pt of ClientH] (Client) {+ name: String\\+ points: int};

\node[header] (SuppH) [left=of ProdH, xshift=-2cm] {Supplier};
\node[class, below=0pt of SuppH] (Supp) {+ name: String\\+ contact: String};

% Relations
\draw[arrow] (Sale) -- (User) node[midway, above, font=\tiny] {effectué par};
\draw[composition] (Sale) -- (Item);
\draw[arrow] (Item) -- (Prod);
\draw[arrow] (Sale) -- (Client);
\draw[arrow] (Prod) -- (Supp);

\end{tikzpicture}
\end{center}

\subsection{Description des Classes}
\begin{itemize}[leftmargin=*]
    \item \textbf{Product} : Représente un médicament ou article en vente. Gère le stock, le prix et le seuil d'alerte.
    \item \textbf{Sale} : Enregistre une transaction de vente. Elle est liée à un client et un utilisateur, et contient une liste d'articles.
    \item \textbf{SaleItem} : Ligne de vente associant un produit, une quantité et un prix unitaire au moment de la vente.
    \item \textbf{Client} : Gère les informations des clients et leurs points de fidélité.
    \item \textbf{Supplier} : Représente les fournisseurs pour le réapprovisionnement.
    \item \textbf{User} : Gère les comptes utilisateurs (Admin, Employé) et leurs droits d'accès.
\end{itemize}

\section{Principales Fonctionnalités Implémentées}
Le système offre une suite complète d'outils pour la gestion quotidienne :

\begin{enumerate}
    \item \textbf{Gestion des Ventes (Caisse)}
    \begin{itemize}
        \item Interface de vente rapide avec ajout de produits au panier.
        \item Calcul automatique du total.
        \item Vérification instantanée du stock disponible (blocage si stock insuffisant).
        \item Association optionnelle à un client fidélisé.
    \end{itemize}

    \item \textbf{Gestion de Stock}
    \begin{itemize}
        \item Ajout, modification et suppression de produits.
        \item Suivi des niveaux de stock et alertes de stock minimal.
    \end{itemize}

    \item \textbf{Tableau de Bord}
    \begin{itemize}
        \item Visualisation des indicateurs clés (Stock critique, Ventes du jour).
        \item Navigation par onglets pour une ergonomie fluide.
    \end{itemize}

    \item \textbf{Administration}
    \begin{itemize}
        \item Authentification sécurisée (Login/Mot de passe).
        \item Gestion des utilisateurs et des rôles.
    \end{itemize}
\end{enumerate}

\section{Modèle Conceptuel de Données (MCD)}
Le MCD illustre les relations sémantiques entre les entités métier.

\begin{center}
\begin{tikzpicture}[
    node distance=2cm,
    entity/.style={rectangle, draw=primary, thick, minimum width=2cm, minimum height=1cm, align=center},
    relation/.style={ellipse, draw=black, minimum width=1.5cm, align=center},
    link/.style={-, thick}
]

\node[entity] (client) {CLIENT};
\node[entity] (vente) [right=of client] {VENTE};
\node[entity] (produit) [below=of vente] {PRODUIT};

\node[relation] (passer) [between=client and vente] {Passer};
\node[relation] (contenir) [between=vente and produit] {Contenir\\(Qté)};

\draw[link] (client) -- node[above] {0,n} (passer);
\draw[link] (passer) -- node[above] {1,1} (vente);
\draw[link] (vente) -- node[left] {1,n} (contenir);
\draw[link] (contenir) -- node[left] {0,n} (produit);

\end{tikzpicture}
\end{center}

\section{Modèle Logique de Données (MLD)}
Règles de transformation : les \textbf{\underline{clés primaires}} sont soulignées, les \textbf{clés étrangères} sont précédées d'un \#.

\begin{itemize}
    \item \textbf{SUPPLIER} (\underline{id\_supplier}, name, contact, email, address)
    
    \item \textbf{USER} (\underline{id\_user}, username, password, role)
    
    \item \textbf{CLIENT} (\underline{id\_client}, name, phone, email, loyalty\_points)
    
    \item \textbf{PRODUCT} (\underline{id\_product}, name, description, price, stock\_quantity, min\_stock\_level, \#id\_supplier)
    
    \item \textbf{SUPPLIER\_ORDER} (\underline{id\_order}, order\_date, status, \#id\_supplier)
    
    \item \textbf{SALE} (\underline{id\_sale}, sale\_date, total\_amount, \#id\_user, \#id\_client)
    
    \item \textbf{CONTIENT\_PRODUIT} (\underline{\#id\_order}, \underline{\#id\_product}, quantity)
    
    \item \textbf{COMPOSE} (\underline{\#id\_sale}, \underline{\#id\_product}, quantity, unit\_price)
\end{itemize}

\section{Difficultés Rencontrées}
Durant le développement, plusieurs défis techniques et organisationnels ont été surmontés :

\begin{description}
    \item[Environnement et Base de Données] Le développement a été réalisé sous Linux. Une incompatibilité avec les serveurs MySQL locaux a nécessité la migration vers MariaDB pour assurer la stabilité du système.
    \item[Apprentissage de JavaFX] N'ayant pas étudié JavaFX en cours, j'ai dû me former en autodidacte pour concevoir l'interface graphique (GUI) et gérer les interactions utilisateurs.
    \item[Utilisation de l'IA (DAO et GUI)] Face à des concepts non couverts en classe, notamment le patron de conception Data Access Object (DAO) et la complexité de JavaFX, j'ai utilisé l'intelligence artificielle pour générer une structure de base propre et comprendre les meilleures pratiques d'implémentation.
\end{description}

\end{document}
